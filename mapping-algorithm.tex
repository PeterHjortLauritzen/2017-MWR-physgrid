\subsection{Mapping between grids}
A consequence of separating physics and dynamics grids is that the atmospheric state must be mapped to the physics grid and the physics tendencies must be mapped back to the dynamics grid. The dynamics state is defined on the GLL grid in terms of temperature $T^{(gll)}$, zonal wind component $u^{(gll)}$, meridional wind component $v^{(gll)}$, dry pressure level thickness $\Delta p^{(gll)}$ and dry tracer mixing ratio $m^{(gll)}$ (the procedure for mapping tracers is the same for all tracers so just one generic tracer is included in the notation/algorithm).



In the mapping of the atmospheric state to the physics grid it is important that the following properties are met:
\begin{enumerate}
\item conservation of scalar quantities such as mass and thermal energy,\label{prop1}
\item for tracers; shape-preservation (monotonicity), i.e. the mapping method must not introduce new extrema in the interpolated field, in particular, negatives,\label{prop2}
\item consistency, i.e. the mapping preserves a constant,\label{prop3}
\item linear correlation preservation.
\end{enumerate}
Other properties that may be important, but not pursued here, is total energy conservation and axial angular momentum conservation. 

The physics tendencies are computed on the finite-volume physics grid and are denoted $f_T^{(phys)}$,$f_u^{(phys)}$,$f_v^{(phys)}$, and $f_m^{(phys)}$. Note that dry air mass is not modified by physics and hence there is no tendency for dry mass,  $f_{\Delta p}\equiv 0$.

and must be mapped back to the GLL grid. It is important that this process:
\begin{enumerate}
\item preserves a zero tendency,
\item for tracers; mass tendency is conserved,
\item linear correlation preservation,
\item consistency, i.e. the mapping preserves a constant.
\end{enumerate}
Other properties that may be important, but not pursued here, is total energy conservation and axial angular momentum conservation. 

It is argued that the most accurate and consistent method to transfer the dynamics state from the GLL grid to the physics grid is to integrate the SE basis function representation (Lagrange polynomials) over the physics grid control volumes. This satisfies property \ref{prop1} and \ref{prop3} but not necessarily \ref{prop2} as the basis functions themselves are oscillatory. We will return to this issue.


Note that tendencies and not an updated state is mapped back to the dynamics grid. If one were to map an updated state the errors in the mapping process may adversely affect the simulation, e.g., in the case of no physics forcing there will be a non-zero `physics forcing' entirely due to the errors in the mapping algorithm.

